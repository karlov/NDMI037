\section{\texorpdfstring{Interval graphs}{Interval graphs}}
\vspace{5mm}
\large

If there is an interval that is \emph{open} on one side, we can replace it by smaller one but closed.
This does not change the graph.
Therefore WLOG we may assume that all intervals are closed.

Works because we have finite number of intervals.

\begin{definition}[Interval graph(INT)]
	\[ INT = IG(\{\text{interval on a line}\}) = IG(\{\text{closed intervals on a line}\}) \]
\end{definition}

\begin{theorem}[Chordal equivalent statements]
	For any graph $G$ the following are equivalent:
	\begin{enumerate}
		\item $G \in INT$
		\item $G$ allows a Clique-path decomposition
		\item $G \in IG(\{\text{subpaths of a path}\})$
	\end{enumerate}
\end{theorem}
\begin{proof}
	$1 \iff 3$
	% TODO picture

	Let $Q$ be a maximal clique of $G$.
	All intervals that are represented by $Q$ intersect.
	\[ \forall u,v \in Q: I_u \cap I_v \ne \emptyset \]
	Intervals on a line have a Helly 2-property.
	E.g. convex sets in the plain have Helly 3-property.

	Therefore
	\[ \bigcap_{u \in Q} I_n \ne \emptyset \]

	% TODO picture
	Let $x$ be the rightmost of the endpoints of $I_u, u \in Q$.
	Then $x \in \bigcap_{u \in Q} I_n$
	As otherwise
	\[ \exists v \in Q: x \notin I_v \]
	Also
	\[ \exists w \in Q: w = [x, \ldots] \Rightarrow I_v \cap I_w = \emptyset \]
	and otherwise $x$ was not the rightmost point.

	$G$ has maximal cliques: $Q = \{ Q_1, \ldots, Q_k \}$.
	Let $p_1, \ldots, p_k$ be path such that
	\[ p_i \in \bigcap_{u \in Q_i} I_u \]
	It cannot happen that $p_i = p_{i + 1}$ as it means $Q_i \cap Q_{i + 1} = p_i$.
	Combining the two will make a larger clique.
	Therefore
	\[ p_1 < \ldots < p_k \]
	Then
	\[ P = (\Q, \{ Q_i, Q_{i + 1} | i = 1, 2, \ldots, k - 1 \}) \]
	is a clique-path decomposition.

	Having
	\[ u \in Q_i, Q_h: i < h \]
	We want for $i < j < h \Rightarrow p_i, p_g \in I_u \Rightarrow p_j \in I_u \Rightarrow q \in Q_j$.

	$1 \iff 2$.

	"$\Rightarrow$"

	\[ uv \in E(G) \Rightarrow \exists i: u,v \in Q_i \Rightarrow Q_i \in V_u \cap V_v \ne \emptyset \]

	"$\Leftarrow$"
	\[ V_u \cap V_v \in Q_i \Rightarrow u,v \in Q_i \Rightarrow  uv \in E(G) \]

\end{proof}

\begin{definition}[Comparability graphs (CO)]
	POSET is $\Po = (P, \leq)$, $\leq$ is
	\begin{enumerate}
		\item antireflexive $x \not \leq x$ - no loops
		\item antisymmetric $x \leq y \Rightarrow y \not \leq x$ - no oriented 2 cycles
		\item transitive $x \leq y \land y \leq z \Rightarrow x \leq z$ - edge between 2 connected
	\end{enumerate}

	$G$ is comparability graph $\iff \exists \Po = (P, \leq)$, such that
	\[ G \simeq (P, \{ xy: x\leq y \lor y \leq x \}) := C_G(\Po) \]

	Comparing to Hasse diagram, comparability graph has more edges as in Hasse diagram we do not depict edges that come from transitivity.
\end{definition}

\begin{observation}
	How to check if graph is comparability?
	Orient every edge from the smaller to larger element.
	$G \in CO \iff$ edges of $G$ can be transitively oriented.

	Every edge gets only 1 orientation.

	Note that such orientation preserves antireflexive and antisymmetric property.
\end{observation}

\begin{notation}[Graph Complement]
	\[ -G = \left(V(G), \binom{V(G)}{2} \setminus E(G)\right) \]

	If $\A$ is a class of graphs, then $co-\A = \{ -G: G \in \A \}$.
\end{notation}

\begin{theorem}[Complements]
	.
	\begin{enumerate}
		\item $FUN = co-CO$
		\item $PER = CO \cap co-CO$
		\item $INT = CHOR \cap co-CO$
	\end{enumerate}
\end{theorem}
\begin{proof}
	$FUN \subseteq co-CO$

	Let $G \in FUN$ and $\{ c_u, u \in V(G) \}$ is a $FUN$ representation.
	Having
	\[ u \ne v: uv \ne E(G) \Rightarrow c_u \cap c_v = \emptyset \]

	From the Jordan path theorem, plane is divided into 2 parts by the line.
	$-G$ can be oriented such that $u \rightarrow v \iff c_u$ lies below $c_v$.

	Therefore either $u \rightarrow v \lor v \rightarrow u$.

	Orientation is transitive from %TODO picture.

	$co-CO \subseteq FUN^{\ast} \subseteq FUN$
	Having $G \in co-CO$ we can transitively orient non edges $\binom{V(G)}{2} \setminus E(G)$.
	Which corresponds to some partial order.

	\begin{definition}[PO dimension]
		Dimension of partial order is minimal number of linear orders $k$ such that
		\[ \exists L_1, \ldots, L_k: \Po = \bigcap L_i \]
	\end{definition}

	Take $k$ vertical lines and make $n$ points on each of the line.
	Points on the line correspond to $V(G)$.
	Put vertices in the order of linear order $L_i$ for line $i$.
	Connect the occurrences of vertex on each line.

	Take $u, v \in V(G)$ and check their lines
	\begin{gather*}
		uv \in E(G) \iff uv \notin E(-G) \Rightarrow uv \notin P \land vu \notin P \\
	\Rightarrow \exists i \ne j uv \in L_i \land vu \in L_j \Rightarrow c_u \cap c_v \ne \emptyset
	\end{gather*}
	On the other hand
	\begin{gather*}
		wv \notin E(G) \Rightarrow uv \in E(-G) \Rightarrow uv \in \Po \lor vu \notin \Po \\
		\Rightarrow \forall i: uw \in L_i \Rightarrow c_u \cap c_w = \emptyset
	\end{gather*}
	Such piecewise continuous lines is the representation of the graph as a FUN graph.

% 3rd lecture
	\begin{lemma}
		\[ co-PER \subseteq PER \]
	\end{lemma}
	\begin{proof}
		Given a representation, swap order of the right side.

		If $uv \in E(G)$ then after transformation edges would not cross.
		Otherwise $uv \notin E(G)$ then edges do not cross but after the transformation edges cross.
	\end{proof}

	\begin{consequence}
		Complements of the permutation graphs are permutation graphs
	\end{consequence}
	\begin{proof}
		if $co-PER \subseteq PER \Rightarrow$:
		\[ PER = co-(co-PER) \subseteq co-PER \]
		Therefore
		\[ PER = co-(co-PER) \]
	\end{proof}

	Proof of 2.
	"$PER \subseteq CO \cap co-CO$" then using lemma and 1. we get
	\[ PER \subseteq FUN = co-CO \]
	\[ PER = co-PER \subseteq co-(co-CO) = CO \]

	"$CO \cap co-CO \subseteq PER$" Let $G$ is a graph, $G \in CO \cap co-CO$.
	$E(G)$ can be transitively oriented, let $E_1$ be such orientation.
	Also, edges of the complements can be transitively oriented, let $E_2$ be such orientation.
	Taking union $E_1 \cup E_2$ of both orientations as binary relations gives orientation on $K_{V(G)}$.
	New relation is \emph{antisymmetric}, \emph{antireflexive} and also \textbf{transitive}.
	Transitivity can be proved by checking all combinations of 2 edges $u_1,u_2 \in E_1, E_2$.
	It is also a linear order, because of the complete graph.

	\begin{observation}[Reverse transitive orientation]
		Note that reverse of the transitive orientation is also transitive.
	\end{observation}

	Draw a linear order $E_1 \cup E_2$ on left line and $E_1^{-1} \cup E_2$ of the right.
	If $uv \in E(G)$ then $uv$ is oriented in $E_1$, on the right line the order is reversed therefore lines cross.
	\[ S_u \cap S_v \ne \emptyset \]
	If $wu \notin E(G)$ it is oriented in $E_2$.
	Therefore lines do not cross.
	\[ S_u \cap S_2 = \emptyset \]

	3. Interval graphs are chordal as they are intersection graphs of subpath (connected subpaths) on a path.
	Which is a subclass if connected subgraphs of trees = Chordal.

	"$INT \subseteq co-Co$". Let $G \in INT$.
	$-G$ is a graph of disjoint intervals.
	Natural transitive orientation of disjoint intervals is "being to the left/right".

	"$CHOR \cap co-Co \subseteq INT$". Suppose $G \in CHOR \cap co-Co$ and $E_2$ is transitive orientation of non-edges of $G$.
	Let $\Q$ be a set of all maximal cliques of $G$.
	Define $Q_i < Q_j \in \Q: \exists u \in Q_i, v \in Q_j: (u \rightarrow v) \in E_2$.
	Properties of relation on cliques:
	\begin{itemize}
		\item antireflexive: as clique is a complete graph and $u\rightarrow v \land v \rightarrow u$ cannot happen.
		\item antisymmetric \\
			Suppose $Q_i < Q_j \Rightarrow \exists u \in Q_i, v \in Q_j: (u \rightarrow v) \in E_2$ and if $Q_j < Q_i$ by contradiction.
			\[ Q_j < Q_i \exists x: Q_i, y \in Q_k: yx \in E_2 \]
			Cases:
			% TODO finish paper
			\begin{enumerate}
				\item $y = v, x \ne v$
				\item $y \ne v, x = v$ same as 1.
			\end{enumerate}
		\item transitive\\
			$Q_i < Q_j < Q_h \Rightarrow Q_i < Q_h$
			\[ \exists x \in Q_i, u,v \in Q_j: \exists y \in Q_H : xu,vy \in E_2 \]
			Cases
			\begin{enumerate}
				\item $u = v \Rightarrow x \rightarrow u \rightarrow y \Rightarrow x \rightarrow y$ by transitivity.
				\item $u \ne v$, subcases
					\begin{enumerate}
						\item $xv \notin E(G) \Rightarrow xv \in E_2 \Rightarrow x \rightarrow y$.
							By transitivity of $E_2$
						\item $uy \notin E(G)$ similar to a)
						\item $xv, uy \in E(G)$, subcases
						\begin{enumerate}
							\item $xv \notin E(G) \Rightarrow yx \in E_2 \stackrel{\text{transitivity}}{\rightarrow} yu \in E_2 \Rightarrow uy \notin E(G)$ contradiction
								Or $xy \in E_2 \Rightarrow Q_i < Q_h$.
							\item $xy \in E(G) \Rightarrow C_4 \leq_I G$ contradiction with Chordal.
						\end{enumerate}
					\end{enumerate}
			\end{enumerate}
		\item transitive\\
	\end{itemize}

	It is also a total order
	\[ \forall i \ne j: Q_i < Q_j \lor Q_j < Q_i \]
	As $Q_i \ne Q_j \Rightarrow u \in Q_i \setminus Q_j$.
	\[ u \notin Q_j \Rightarrow v \in Q_j: uv \notin E(G) \]
	As otherwise, clique is not maximal.
	Then $uv$ is oriented one way or another.

	Take $Q_i$ as vertices and make edges $Q_{i}Q{i - 1}$.
	We claim that this graph is a clique path for $G$.

	It remains to prove:
	\[ \forall i < j < h \forall u \in V(G): u \in Q_i \land Q_h \Rightarrow u \in Q_j \]
	Suppose $u \notin Q_j \Rightarrow \exists v \in Q_j: uv \notin E(G)$
	\begin{gather*}
		Q_i < Q_j \Rightarrow uv \in E_2 \\
		Q_j < Q_h \Rightarrow vu \in E_2
	\end{gather*}
	Contradiction with antisymmetric relation.
\end{proof}

\subsection{Interval Filament graphs}

\begin{definition}[Interval Filament graph(IFA)]
	\[ IFA = \IG(\text{interval filaments on the half plane}) \]

	Were introduced by F.Gavril (2000) as an attempt to generalize graphs for which we can find maximal clique in polynomial time.
\end{definition}

\begin{observation}
	\begin{itemize}
		\item $INT \subseteq IFA$ we can draw interval as a half rectangle.
		\item $CHOR \subseteq IFA$
		\item $PER \subseteq IFA$
		\item $FUN \subseteq IFA$ extend the intervals to new line placed below every point of the representation.
		\item $CIR \subseteq IFA$
		\item $CA \subseteq IFA$
		\item $PC \subseteq IFA$
	\end{itemize}
\end{observation}

\begin{exercise}
	Prove that $CHO \subseteq PC \Rightarrow CHO \subseteq IFA$
\end{exercise}

\begin{definition}[$\A$-mixed graphs]
	If $\A$ is a class of graphs, $\A$-mixed graph if it allowes partition $E = E_1 \dot{\cup} E_2$ and a transitive orientation such that
	\begin{enumerate}
		\item $(V, E) \in \A$
		\item $\forall u, v, w \in uv \in E_2 \land vw \in E_1 \Rightarrow uw \in E_1$
	\end{enumerate}
\end{definition}

\begin{theorem}[Complements]
	\[ co-IFA = (co-INT)-mixed \]
\end{theorem}
\begin{proof}
\end{proof}
