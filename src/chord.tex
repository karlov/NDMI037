\section{\texorpdfstring{Chordal graphs}{Chordal graphs}}
\vspace{5mm}
\large

\begin{definition}[Chordal graph]
	G is chordal if $\forall k \geq 4: C_k \not\leq_{ind} G$.
	Sometimes called triangulated graphs.
\end{definition}

\begin{definition}[Simplicial]
	A vertex $u \in V(G)$ is simplicial  if $G[N_G(u)]$ (reduction of graph to neighborhood of $u$) is a complete graph.
	Definition is independent from taking closed (includes $u$) or open neighborhood.
\end{definition}

\begin{lemma}[1]
	Every inclusion-wise minimal vertex cut in a chordal graph induces a clique.
\end{lemma}
\begin{proof}
	$G\setminus A$ has components $V_1, V_2, \ldots V_k, k \geq 2$.
	Then
	\[ \forall i \forall u \in A \exists w \in V_i: uw \in E(G) \]
	Pick some component $V_i$ and some edge in $A$ then there is an edge between them.
	On the contrary, if there is no edge, $u$ can be removed from $A$.
	Which contradicts with minimality of $A$.

	% TODO picture
	Now we take $u, v \in A$, by previous observation
	\[ \exists w_1, w_2 \in V_i: uw_1, vw_2 \in E(G) \]
	Then take $P_1$ shortest path between $w_1, w_2$.
	Similarly $w_3, w_4 \in V_j$ and the shortest path $P_2$ between $w_3, w_4$.

	$P_1 \cup P_2$ is an induced cycle unless $uv \in E(G)$.

	Also, there is no edge between $V_i$ and $V_j, i \ne j$ as otherwise $A$ is not a cut.

	As $P_1$ is shortest path $vw_1, uw_2 \notin E(G)$.

	To sum up, $uv \in E(G)$.
	Since $u,v$ were arbitrary, $A$ is a complete subgraph.
\end{proof}

\begin{lemma}[2]\label{chordal_lemma_2}
	A chordal graph is complete or it contains 2 non-adjacent simplicial vertices.
\end{lemma}
\begin{proof}
	By induction on $|V(G)|$.
	The first step is $G$ is a complete graph.

	Inductive step: $G$ is not complete.
	Take a minimal vertex cut $A$.
	Let $B$ be a connected component of $G\setminus A$ and $C = (V(G) \setminus A) \setminus B$.
	\[ G_1 = G[B \cup A] \]
	\[ G_2 = G[C \cup A] \]

	% TODO picture

	As $|V(G_1)| < |V(G)|$ we can apply induction on it.
	Note that induced subgraph of chordal graph is also chordal.
	By induction hypothesis $G_1, G_2$ are either complete or have 2 simplicial vertices.

	One of the simplicial vertices cannot be in $A$ because $A$ is complete graph and simplicial vertices are not adjacent.
	No edges can connect $B, C$ therefore both of the vertices are simplicial in $G$.
\end{proof}

\begin{corollary}
	Every nonempty chordal graphs has a simplicial vertex.
\end{corollary}

Sometimes it is easier to proof stronger statement, because we have more power in inductive step.

\begin{definition}[PES]
	Perfect elimination scheme - for graph $G$ is a \emph{linear ordering} of its vertices.
	\[ V(G) = u_1, \ldots, u_n\]
	Such that $\forall i: u_i$ is simplicial for $G[\{ u_1, \ldots u_i\}]$
\end{definition}

\begin{lemma}[Chordal has PES]
	Every chordal graph allows a PES.
\end{lemma}
\begin{proof}
	Take any simplicial vertices and move it to the right.
	Then delete vertex picked in previous step and repeat.

	Formally: by induction on $n$ using corollary.
\end{proof}

\begin{definition}[Perfect graph]
	$G$ is perfect if for every subgraph chromatic number is equal to clique number.
	\[ \forall H \leq G: \chi(H) = \omega(H) \]
\end{definition}

\begin{theorem}[Chordal is Perfect]
	A chordal graph is perfect.
\end{theorem}
\begin{proof}
	% TODO rewrite
	Take a PES for a graph, color from left to right by colors $ \in \{ 1, 2, 3, \ldots \}$ by \textbf{first fit method}.

	Take smallest number that was not taken by the neighbors.

	If we a forced to use color $k$ then neighbors of the vertex used $(k - 1)$ colors.
	Which implies existence of complete graph on $(k - 1)$ vertices on the left from current vertex.
\end{proof}

\begin{definition}[Clique-tree decomposition]
	Clique-tree decomposition of a graph $G$ is a tree $T$
	\[ T = (\Q, F): V(T) = \Q = \{\text{maximal cliques of G} \} \]
	and
	\[ \forall u \in V(G): T[ \{Q: u \in Q \in \Q \}] \text{ is connected} \]
\end{definition}

\begin{theorem}[Chordal equivalent statements]
	For any graph $G$ the following are equivalent:
	\begin{enumerate}
		\item $G$ is chordal
		\item $G$ has a PES
		\item $G$ allows a Clique-tree decomposition
		\item $G$ is an intersection graph of subtrees of a tree
	\end{enumerate}
\end{theorem}
\begin{proof}
	$1 \Rightarrow 2$ By induction on the number of vertices, using \cref{chordal_lemma_2}.
	Pick simplicial vertex, put it at the end of PES.
	Remove vertex from graph and continue.

	$2 \Rightarrow 3$ let we have a PES $\{ u_1, \ldots u_n \}$.
	$G$ has maximal cliques: $Q = \{ Q_1, \ldots, Q_k \}$, $T = (\Q, E(T))$.
	Remove last vertex in PES from graph
	\[ G^{'} = G \setminus u_n \]
	It has cliques: $Q^{'} = \{ Q_1^{'}, \ldots, Q_k^{'} \} \Rightarrow$ by i.h
	\[ \exists T^{'} = (\Q^{'}, E(T^{'}) \]
	Consider 2 cases: $N_G(u_n) \cup \{ u_n \}$ is a maximal clique.
	Then is a maximal clique in $G^{'}$.

	O/w $N_G(u_n)$ is not a maximal clique of $G^{'}$.
	$\Rightarrow \exists Q_i^{'} \in \Q^{'}: N_G(u_n) \subsetneq Q_i^{'}$.
	We connect $N_G(u_n)$ to $Q_i^{'}$.

	$3 \Rightarrow 4$ we want to find an intersection graph of subtrees in tree.
	\[ G \simeq IF(\{T_u: u \in V(G) \}) \]
	such that
	\[ V(T_u) = \{ Q_i: y \in Q_i \} \subseteq \Q \]

	Proving the equivalence:
	\[ uv \in E(G) \Rightarrow \exists Q_i \in \Q: u,v \in Q_i \Rightarrow Q_i \in V(T_u) \cap V(T_v) \neq \emptyset \]
	On the other hand
	\[ V(T_u) \cap V(T_v) \neq \emptyset \Rightarrow \exists Q_i \in V(T_u) \cap V(T_v) \Rightarrow u, v \in Q_i \Rightarrow uv \in E(G) \]

	$4 \Rightarrow 1$ Let we have a tree $T$ with a collection of subtrees.
	\[V_u \subseteq V(T), u \in V(G): T[V_u] \text{ is connected}: \forall u \ne v \in V(G): uv \in E(G) \iff V_u \cap V_v \ne \emptyset \]
	Assume by contradiction $G$ is not chordal.
	By definition $\exists k \geq 4 \in \N: C_k \leq_{ind} G$.
	Let the cycle be $\{ u_1, u_2, \ldots, u_k \}$.
	$u_k u_1 \in E(G)$ by the assumption, take
	\[ T_1 = T[V_{u_1}], T_2 = T[V_{u_2}], T_3 = T[V_{u_3}] \]
	$T_1$ should cross $T_2$, also $T_3$ should cross $T_2$ but not $T_1$.
	\[ V_{u_1} \cap V_{u_2} \ne \emptyset \land V_{u_1} \cap V_{u_3} = \emptyset \]
	Therefore
	\[ \exists e \in E(T_2): e \notin E(T_1), E(T_2) \]
	Removing 1 edge makes tree disconnect:
	% TODO disjoint union
	\[ T \setminus e = T_a \mathbin{\dot{\cup}} T_b, V_{u_1} \subseteq V(T_a), V_{u_3} \subseteq V(T_b) \]
	Proceed by induction
	\[ \forall j, j = 3 \ldots k: V(T_j) \subseteq T_b \]
	inductive step
	\[ V_{u_j} \subseteq V(T_b), V_{u_{j + 1}} \cap V(T_b) \neq \emptyset, V_{u_{j + 1}} \cap V_{u_2} \ne \emptyset \Rightarrow V_{u_{j + 1}} \subseteq V(T_b)\]
	Therefore
	\[ V_{u_k} \cap V_{u_1} = \emptyset \]
	which contradicts $u_k u_1 \in E(g)$.

	% TODO next lecture

\end{proof}
