\section{\texorpdfstring{Introduction}{Introduction}}
\vspace{5mm}
\large

\subsection{Course overview}

\begin{itemize}
	\item Finite graphs
	\item Existence theorems
	\item Algorithms, NP-completeness
	\item Graph drawing
\end{itemize}

How much information do we need to store the drawing?

Upward drawing: each edge directed upwards.

Most of the time we talk about \textbf{Intersection graphs}.

\subsection{Intersection graphs}

\begin{definition}[Intersection graph]
	Let $\A$ be a family of sets.
	Then \textbf{the} intersection graph is
	\[ IG(\A) = (\A, \{ ab: a\ne b, a \cap b \ne \emptyset, a, b \in \A \}) \]

	Let $\M$ be a large family of sets, then G is \textbf{an} intersection graph of $\M$ if
	\[ \exists A \subseteq \M: G \simeq IG(A) \]

	Note that in a family an element can be repeated several times.
\end{definition}

\begin{observation}
	IF $\M$ contains nonempty set, then $\forall$ complete graphs is in $\IG(\M)$.
\end{observation}
\begin{proof}
	\begin{gather*}
		A \in \M, A \ne \emptyset \\
		V(K_n) = \{ u_1, u_2, \ldots, u_m \} \\
	\end{gather*}
	Every vertex $u$ is represented by $A$.
\end{proof}

\begin{observation}
	\[ G \in \IG(\M) \iff \exists f:V(G) \to \M: \forall u \ne v \in V(G): uv \in E(G) \iff f(n) \cap f(v) \ne \emptyset \]
\end{observation}

\begin{observation}
	$\forall \M: \IG(M)$ is \textbf{hereditary} if
	\[ \forall G \in \IG(\M) \ \forall H \leq G: H \in IG(\M) \]

	Induced subgraph is just deleting some edges, which in sets case means forgetting edges that represent sets.
\end{observation}

\begin{definition}[Interval graph]
	% TODO Picture
	Are intersection graphs of intervals on connected subsets of 1 dimensional Euclidian space.
\end{definition}

We are interested in \emph{arc connected} sets of the plain.
\begin{definition}[Arc connected]
	% TODO

	Is different from topological connected.
\end{definition}

\begin{definition}[Circle graphs (CA)]
	Arcs on a circle.
\end{definition}

\begin{definition}[Circular arc graphs (CIR)]
	Chords on a circle.
\end{definition}

\begin{definition}[Polygon circle graphs (PC)]
	Polygons that can be inscribed in a circle.
\end{definition}

\begin{definition}[Permutation graphs (PER)]
	Segments connecting two parallel lines.

	Formally:
	\begin{gather*}
		V(G) = \{ u_1, u_2, \ldots, u_m \} \\
		\exists \pi \in Sym(m): u,v \in E \iff i < j \land \pi(i) > \pi (j)
	\end{gather*}
\end{definition}

\begin{definition}[Function graphs (FUN)]
	Curves connection two parallel lines.
\end{definition}

\begin{definition}[Segments graphs (SEG)]
	Straight-line segments in the plane.
\end{definition}

\begin{definition}[String graphs (STRING)]
	Curves in the plane.
\end{definition}

% TODO finish, statement
For each 2 sets, take a point that lies in the intersection of these sets.
Then connect unused dots by branching curve.

Problems we want to solve:
\begin{enumerate}
	\item Given a graph, does it belong to the class.
	\item How can we describe a representation. What size?
	\item Given a graph and a representation, can we find max independent set, clique and so on.
	Such questions that are NP-complete in general.
\end{enumerate}

Q: TODO is graph class recognition decidable?\\
A: Yes, but not polynomial in all cases.

INT, CA, CIR, PC, PER, FUN can be polynomially recognized.
SEG, CONN are considered NP-hard.
STRING is NP-complete.

\[ \forall n \exists G_N \in SEG\]
but in $\forall SEG$ segment representation there is a coordinate that is at least double exponential $2^{2^n}$, so even in bits are exponential.

Representation of STRING graph: make each intersection as a vertex.
All curves between intersection are edges.
Then graph becomes planar.
Planarity can be checked in Polynomial time.

\[ \forall n \exists G_n \in STRING \forall \text{representation requires} \geq 2^{cn} \text{crossing points} \]

\subsection{Chordal graphs}

\begin{definition}[Chordal graph]
	G is chordal if $\forall k \geq 4: C_k \not\leq G$.
	Sometimes called triangulated graphs.
\end{definition}

\begin{definition}[Simplicial]
	A vertex $u \in V(G)$ is simplicial  if $G[N_G(u)]$ (reduction of graph to  is a complete graph.
	Definition is independent from taking closed (includes $u$) or open neighborhood.
\end{definition}

\begin{lemma}[1]
	Every inclusion-vise minimal vertex cut in a chordal graph induces a clique.
\end{lemma}
\begin{proof}
	$G\setminus A$ has components $V_1, V_2, \ldots V_k, k \geq 2$.
	Then
	\[ \forall i \forall u \in A \exists w \in V_i: uw \in E(G) \]
	Pick some component $V_i$ and some edge in $A$ then there is an edge between them.
	On the contrary, if there is no edge, $u$ can be removed from $A$.
	Which contradicts with minimality of $A$.

	% TODO picture
	Now we take $u, v \in A$, by previous observation
	\[ \exists w_1, w_2 \in V_i: uw_1, vw_2 \in E(G) \]
	Then take $P_1$ shortest path between $w_1, w_2$.
	Similarly $w_3, w_4 \in V_j$ and the shortest path $P_2$ between $w_3, w_4$.

	$P_1 \cup P_2$ is an induced cycle unless $uv \in E(G)$.

	Also, there is no edge between $V_i, V_j, i \ne j$ as o/w $A$ is not a cut.

	As $P_1$ is shortest path $vw_1, uw_2 \notin E(G)$.

	To sum up, $uv \in E(G)$.
	Since $u,v$ were arbitrary, $A$ is a complete subgraph.
\end{proof}

\begin{lemma}[2]\label{chordal_lemma_2}
	A chordal graph is complete or it contains 2 non-adjacent simplicial vertices.
\end{lemma}
\begin{proof}
	By induction on $|V(G)|$.
	The first step is $G$ is a complete graph.

	Inductive step: $G$ is not complete.
	Take a minimal vertex cut $A$.
	Let $B$ be a connected component of $G\setminus A$ and $C = (V(G) \setminus A) \setminus B$.
	\[ G_1 = G[B \cup A] \]
	\[ G_2 = G[C \cup A] \]

	% TODO picture

	As $|V(G_1)| < |V(G)|$ we can apply induction on it.
	Note that induced subgraph of chordal graph is also chordal.
	By induction hypothesis $G_1, G_2$ are either complete or have 2 simplicial vertices.

	One of the simplicial vertices cannot be in $A$ because $A$ is complete graph and simplicial vertices are not adjacent.
	No edges can connect $B, C$ therefore both of the vertices are simplicial in $G$.
\end{proof}

\begin{corollary}
	Every nonempty chordal graphs has a simplicial vertex.
\end{corollary}

Sometimes it is easier to proof stronger statement, because we have more power in inductive step.

\begin{definition}[PES]
	Perfect elimination scheme - for graph $G$ is a \emph{linear ordering} of its vertices.
	\[ V(G) = u_1, \ldots, u_n\]
	Such that $\forall i: u_i$ is simplicial for $G[\{ u_1, \ldots u_i\}]$
\end{definition}

\begin{note}
	Every chordal graph allows a PES.
\end{note}
\begin{proof}
	Take any simplicial vertices and move it to the right.
	Then delete vertex picked in previous step and repeat.

	Formally: by induction on n using corollary.
\end{proof}

\begin{definition}[Perfect graph]
	G is perfect if for every subgraph chromatic number is equal to clique number.
	\[ \forall H \leq G: \chi(H) = \omega(H) \]
\end{definition}

\begin{theorem}[Chordal is Perfect]
	A chordal graph is perfect.
\end{theorem}
\begin{proof}
	Take a PES for a graph, color from left to right by colors which a numbers
	\[ \{ 1, 2, 3, \ldots \} \]
	by \textbf{first fit method}.

	Take smallest number that was not taken by the neighbors.

	If we a forced to use color $k$ then neighbors of the vertex used $(k - 1)$ colors.
	Meaning that on the left there is complete graph on $(k - 1$ vertices.
\end{proof}

\begin{definition}[Clique-tree decompositon]
	Clique-tree decomposition of a graph $G$ is a tree $T$
	\[ T = (\Q, F): V(T) = \Q = \{\text{maximal cliques of G} \} \]
	and
	\[ \forall u \in V(G): T[ \{Q: u \in Q \in \Q \}] \text{ is connected} \]
\end{definition}

\begin{theorem}[Chordal equivalent statements]
	For any graph $G$ the following are equivalent:
	\begin{enumerate}
		\item $G$ is chordal
		\item $G$ has a PES
		\item G allows a Clique-tree decomposition
		\item G is an intersection graph of subtrees of a tree
	\end{enumerate}
\end{theorem}
\begin{proof}
	$1 \Rightarrow 2$ By induction on the number of vertices, using \cref{chordal_lemma_2}.
	% TODO next lecture

\end{proof}
