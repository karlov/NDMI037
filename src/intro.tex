\section{\texorpdfstring{Introduction}{Introduction}}
\vspace{5mm}
\large

\subsection{Course overview}

\begin{itemize}
	\item Finite graphs
	\item Existence theorems
	\item Algorithms, NP-completeness
	\item Graph drawing
\end{itemize}

How much information do we need to store the drawing?

Upward drawing: each edge directed upwards.

Most of the time we talk about \textbf{Intersection graphs}.

\subsection{Intersection graphs}

\begin{definition}[Intersection graph]
	Let $\A$ be a family of sets.
	Then \textbf{the} intersection graph is
	\[ IG(\A) = (\A, \{ ab: a\ne b, a \cap b \ne \emptyset, a, b \in \A \}) \]

	Let $\M$ be a large family of sets, then G is \textbf{an} intersection graph of $\M$ if
	\[ \exists A \subseteq \M: G \simeq IG(A) \]

	Note that in a family an element can be repeated several times.
\end{definition}

\begin{observation}
	IF $\M$ contains nonempty set, then $\forall$ complete graphs is in $\IG(\M)$.
\end{observation}
\begin{proof}
	\begin{gather*}
		A \in \M, A \ne \emptyset \\
		V(K_n) = \{ u_1, u_2, \ldots, u_m \} \\
	\end{gather*}
	Every vertex $u$ is represented by $A$.
\end{proof}

\begin{observation}
	\[ G \in \IG(\M) \iff \exists f:V(G) \to \M: \forall u \ne v \in V(G): uv \in E(G) \iff f(n) \cap f(v) \ne \emptyset \]
\end{observation}

\begin{observation}
	$\forall \M: \IG(M)$ is \textbf{hereditary} if
	\[ \forall G \in \IG(\M) \ \forall H \leq G: H \in IG(\M) \]

	Induced subgraph is just deleting some edges, which in sets case means forgetting edges that represent sets.
\end{observation}

\begin{definition}[Interval graph]
	% TODO Picture
	Are intersection graphs of intervals on connected subsets of 1 dimensional Euclidian space.
\end{definition}

We are interested in \emph{arc connected} sets of the plain.
\begin{definition}[Arc connected]
	% TODO

	Is different from topological connected.
\end{definition}

\begin{definition}[Circle graphs (CA)]
	Arcs on a circle.
\end{definition}

\begin{definition}[Circular arc graphs (CIR)]
	Chords on a circle.
\end{definition}

\begin{definition}[Polygon circle graphs (PC)]
	Polygons that can be inscribed in a circle.
\end{definition}

\begin{definition}[Permutation graphs (PER)]
	Segments connecting two parallel lines.

	Formally:
	\begin{gather*}
		V(G) = \{ u_1, u_2, \ldots, u_m \} \\
		\exists \pi \in Sym(m): u,v \in E \iff i < j \land \pi(i) > \pi (j)
	\end{gather*}
	% TODO check
\end{definition}

\begin{definition}[Function graphs (FUN)]
	Curves connecting two parallel lines.
\end{definition}

\begin{definition}[Segments graphs (SEG)]
	Straight-line segments in the plane.
\end{definition}

\begin{definition}[String graphs (STRING)]
	Curves in the plane.
\end{definition}

% TODO finish, statement
For each 2 sets, take a point that lies in the intersection of these sets.
Then connect unused dots by branching curve.

Problems we want to solve:
\begin{enumerate}
	\item Given a graph, does it belong to the class.
	\item How can we describe a representation. What size?
	\item Given a graph and a representation, can we find max independent set, clique and so on.
	Such questions that are NP-complete in general.
\end{enumerate}

Q: TODO is graph class recognition decidable?\\
A: Yes, but not polynomial in all cases.

INT, CA, CIR, PC, PER, FUN can be polynomially recognized.
SEG, CONN are considered NP-hard.
STRING is NP-complete.

\[ \forall n \exists G_N \in SEG\]
but in $\forall SEG$ segment representation there is a coordinate that is at least double exponential $2^{2^n}$, so even in bits is exponential.

Representation of STRING graph: make each intersection as a vertex.
All curves between intersection are edges.
Then graph becomes planar.
Planarity can be checked in Polynomial time.
But also $\forall n \exists G_n \in $ STRING $\forall$ representation requires $\geq 2^{cn}$ crossing points.
