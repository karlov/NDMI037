\section{\texorpdfstring{Introduction}{Introduction}}
\vspace{5mm}
\large

\subsection{Course overview}

\begin{itemize}
	\item Finite graphs
	\item Existence theorems
	\item Algorithms, NP-completeness
	\item Graph drawing
\end{itemize}

How much information do we need to store the drawing?

Upward drawing: each edge directed upwards.

Most of the time we talk about \textbf{Intersection graphs}.

\subsection{Intersection graphs}

\begin{definition}[Intersection graph]
	Let $\A$ be a family of sets.
	Then \textbf{the} intersection graph is
	\[ IG(\A) = (\A, \{ ab: a\ne b, a \cap b \ne \emptyset, a, b \in \A \}) \]

	Let $\M$ be a large family of sets, then G is \textbf{an} intersection graph of $\M$ if
	\[ \exists A \subseteq \M: G \simeq IG(A) \]

	Note that in a family an element can be repeated several times.
\end{definition}

\begin{observation}
	IF $\M$ contains nonempty set, then $\forall$ complete graphs is in $\IG(\M)$.
\end{observation}
\begin{proof}
	\begin{gather*}
		A \in \M, A \ne \emptyset \\
		V(K_n) = \{ u_1, u_2, \ldots, u_m \} \\
	\end{gather*}
	Every vertex $u$ is represented by $A$.
\end{proof}

\begin{observation}
	\[ G \in \IG(\M) \iff \exists f:V(G) \to \M: \forall u \ne v \in V(G): uv \in E(G) \iff f(n) \cap f(v) \ne \emptyset \]
\end{observation}

\begin{observation}
	$\forall \M: \IG(M)$ is \textbf{hereditary} if
	\[ \forall G \in \IG(\M) \ \forall H \leq G: H \in IG(\M) \]

	Induced subgraph is just deleting some edges, which in sets case means forgetting edges that represent sets.
\end{observation}

\begin{definition}[Interval graph]
	% TODO Picture
	Are intersection graphs of intervals on connected subsets of 1 dimensional Euclidian space.
\end{definition}

We are interested in \emph{arc connected} sets of the plain.
\begin{definition}[Arc connected]
	% TODO

	Is different from topological connected.
\end{definition}

\begin{definition}[Circle graphs (CA)]
	Arcs on a circle.
\end{definition}

\begin{definition}[Circular arc graphs (CIR)]
	Chords on a circle.
\end{definition}

\begin{definition}[Polygon circle graphs (PC)]
	Polygons that can be inscribed in a circle.
\end{definition}

\begin{definition}[Permutation graphs (PER)]
	Segments connecting two parallel lines.

	Formally:
	\begin{gather*}
		V(G) = \{ u_1, u_2, \ldots, u_m \} \\
		\exists \pi \in Sym(m): u,v \in E \iff i < j \land \pi(i) > \pi (j)
	\end{gather*}
	% TODO check
\end{definition}

\begin{definition}[Function graphs (FUN)]
	Curves connecting two parallel lines.
\end{definition}

\begin{definition}[Segments graphs (SEG)]
	Straight-line segments in the plane.
\end{definition}

\begin{definition}[String graphs (STRING)]
	Curves in the plane.
\end{definition}

% TODO finish, statement
For each 2 sets, take a point that lies in the intersection of these sets.
Then connect unused dots by branching curve.

Problems we want to solve:
\begin{enumerate}
	\item Given a graph, does it belong to the class.
	\item How can we describe a representation. What size?
	\item Given a graph and a representation, can we find max independent set, clique and so on.
	Such questions that are NP-complete in general.
\end{enumerate}

Q: TODO is graph class recognition decidable?\\
A: Yes, but not polynomial in all cases.

INT, CA, CIR, PC, PER, FUN can be polynomially recognized.
SEG, CONN are considered NP-hard.
STRING is NP-complete.

\[ \forall n \exists G_N \in SEG\]
but in $\forall SEG$ segment representation there is a coordinate that is at least double exponential $2^{2^n}$, so even in bits is exponential.

Representation of STRING graph: make each intersection as a vertex.
All curves between intersection are edges.
Then graph becomes planar.
Planarity can be checked in Polynomial time.
But also $\forall n \exists G_n \in $ STRING $\forall$ representation requires $\geq 2^{cn}$ crossing points.

\section{Chordal graphs}

\begin{definition}[Chordal graph]
	G is chordal if $\forall k \geq 4: C_k \not\leq G$.
	Sometimes called triangulated graphs.
\end{definition}

\begin{definition}[Simplicial]
	A vertex $u \in V(G)$ is simplicial  if $G[N_G(u)]$ (reduction of graph to neighborhood of $u$) is a complete graph.
	Definition is independent from taking closed (includes $u$) or open neighborhood.
\end{definition}

\begin{lemma}[1]
	Every inclusion-wise minimal vertex cut in a chordal graph induces a clique.
\end{lemma}
\begin{proof}
	$G\setminus A$ has components $V_1, V_2, \ldots V_k, k \geq 2$.
	Then
	\[ \forall i \forall u \in A \exists w \in V_i: uw \in E(G) \]
	Pick some component $V_i$ and some edge in $A$ then there is an edge between them.
	On the contrary, if there is no edge, $u$ can be removed from $A$.
	Which contradicts with minimality of $A$.

	% TODO picture
	Now we take $u, v \in A$, by previous observation
	\[ \exists w_1, w_2 \in V_i: uw_1, vw_2 \in E(G) \]
	Then take $P_1$ shortest path between $w_1, w_2$.
	Similarly $w_3, w_4 \in V_j$ and the shortest path $P_2$ between $w_3, w_4$.

	$P_1 \cup P_2$ is an induced cycle unless $uv \in E(G)$.

	Also, there is no edge between $V_i$ and $V_j, i \ne j$ as otherwise $A$ is not a cut.

	As $P_1$ is shortest path $vw_1, uw_2 \notin E(G)$.

	To sum up, $uv \in E(G)$.
	Since $u,v$ were arbitrary, $A$ is a complete subgraph.
\end{proof}

\begin{lemma}[2]\label{chordal_lemma_2}
	A chordal graph is complete or it contains 2 non-adjacent simplicial vertices.
\end{lemma}
\begin{proof}
	By induction on $|V(G)|$.
	The first step is $G$ is a complete graph.

	Inductive step: $G$ is not complete.
	Take a minimal vertex cut $A$.
	Let $B$ be a connected component of $G\setminus A$ and $C = (V(G) \setminus A) \setminus B$.
	\[ G_1 = G[B \cup A] \]
	\[ G_2 = G[C \cup A] \]

	% TODO picture

	As $|V(G_1)| < |V(G)|$ we can apply induction on it.
	Note that induced subgraph of chordal graph is also chordal.
	By induction hypothesis $G_1, G_2$ are either complete or have 2 simplicial vertices.

	One of the simplicial vertices cannot be in $A$ because $A$ is complete graph and simplicial vertices are not adjacent.
	No edges can connect $B, C$ therefore both of the vertices are simplicial in $G$.
\end{proof}

\begin{corollary}
	Every nonempty chordal graphs has a simplicial vertex.
\end{corollary}

Sometimes it is easier to proof stronger statement, because we have more power in inductive step.

\begin{definition}[PES]
	Perfect elimination scheme - for graph $G$ is a \emph{linear ordering} of its vertices.
	\[ V(G) = u_1, \ldots, u_n\]
	Such that $\forall i: u_i$ is simplicial for $G[\{ u_1, \ldots u_i\}]$
\end{definition}

\begin{lemma}[Chordal has PES]
	Every chordal graph allows a PES.
\end{lemma}
\begin{proof}
	Take any simplicial vertices and move it to the right.
	Then delete vertex picked in previous step and repeat.

	Formally: by induction on n using corollary.
\end{proof}

\begin{definition}[Perfect graph]
	$G$ is perfect if for every subgraph chromatic number is equal to clique number.
	\[ \forall H \leq G: \chi(H) = \omega(H) \]
\end{definition}

\begin{theorem}[Chordal is Perfect]
	A chordal graph is perfect.
\end{theorem}
\begin{proof}
	% TODO rewrite
	Take a PES for a graph, color from left to right by colors $ \in \{ 1, 2, 3, \ldots \}$ by \textbf{first fit method}.

	Take smallest number that was not taken by the neighbors.

	If we a forced to use color $k$ then neighbors of the vertex used $(k - 1)$ colors.
	Which implies existence of complete graph on $(k - 1)$ vertices on the left from current vertex.
\end{proof}

\begin{definition}[Clique-tree decomposition]
	Clique-tree decomposition of a graph $G$ is a tree $T$
	\[ T = (\Q, F): V(T) = \Q = \{\text{maximal cliques of G} \} \]
	and
	\[ \forall u \in V(G): T[ \{Q: u \in Q \in \Q \}] \text{ is connected} \]
\end{definition}

\begin{theorem}[Chordal equivalent statements]
	For any graph $G$ the following are equivalent:
	\begin{enumerate}
		\item $G$ is chordal
		\item $G$ has a PES
		\item $G$ allows a Clique-tree decomposition
		\item $G$ is an intersection graph of subtrees of a tree
	\end{enumerate}
\end{theorem}
\begin{proof}
	$1 \Rightarrow 2$ By induction on the number of vertices, using \cref{chordal_lemma_2}.
	Pick simplicial vertex, put it at the end of PES.
	Remove vertex from graph and continue.

	$2 \Rightarrow 3$ let we have a PES $\{ u_1, \ldots u_n \}$.
	$G$ has maximal cliques: $Q = \{ Q_1, \ldots, Q_k \}$, $T = (\Q, E(T))$.
	Remove last vertex in PES from graph
	\[ G^{'} = G \setminus u_n \]
	It has cliques: $Q^{'} = \{ Q_1^{'}, \ldots, Q_k^{'} \} \Rightarrow$ by i.h
	\[ \exists T^{'} = (\Q^{'}, E(T^{'}) \]
	Consider 2 cases: $N_G(u_n) \cup \{ u_n \}$ is a maximal clique.
	Then is a maximal clique in $G^{'}$.

	O/w $N_G(u_n)$ is not a maximal clique of $G^{'}$.
	$\Rightarrow \exists Q_i^{'} \in \Q^{'}: N_G(u_n) \subsetneq Q_i^{'}$.
	We connect $N_G(u_n)$ to $Q_i^{'}$.

	$3 \Rightarrow 4$ we want to find an intersection graph of subtrees in tree.
	\[ G \simeq IF(\{T_u: u \in V(G) \}) \]
	such that
	\[ V(T_u) = \{ Q_i: y \in Q_i \} \subseteq \Q \]

	Proving the equivalence:
	\[ uv \in E(G) \Rightarrow \exists Q_i \in \Q: u,v \in Q_i \Rightarrow Q_i \in V(T_u) \cap V(T_v) \neq \emptyset \]
	On the other hand
	\[ V(T_u) \cap V(T_v) \neq \emptyset \Rightarrow \exists Q_i \in V(T_u) \cap V(T_v) \Rightarrow u, v \in Q_i \Rightarrow uv \in E(G) \]

	$4 \Rightarrow 1$ Let we have a tree $T$ with a collection of subtrees.
	\[V_u \subseteq V(T), u \in V(G): T[V_u] \text{ is connected}: \forall u \ne v \in V(G): uv \in E(G) \iff V_u \cap V_v \ne \emptyset \]
	Assume by contradiction $G$ is not chordal.
	By definition $\exists k \geq 4 \in \N: C_k \leq_{ind} G$.
	Let the cycle be $\{ u_1, u_2, \ldots, u_k \}$.
	$u_k u_1 \in E(G)$ by the assumption, take
	\[ T_1 = T[V_{u_1}], T_2 = T[V_{u_2}], T_3 = T[V_{u_3}] \]
	$T_1$ should cross $T_2$, also $T_3$ should cross $T_2$ but not $T_1$.
	\[ V_{u_1} \cap V_{u_2} \ne \emptyset \land V_{u_1} \cap V_{u_3} = \emptyset \]
	Therefore
	\[ \exists e \in E(T_2): e \notin E(T_1), E(T_2) \]
	Removing 1 edge makes tree disconnect:
	% TODO disjoint union
	\[ T \setminus e = T_a \mathbin{\dot{\cup}} T_b, V_{u_1} \subseteq V(T_a), V_{u_3} \subseteq V(T_b) \]
	Proceed by induction
	\[ \forall j, j = 3 \ldots k: V(T_j) \subseteq T_b \]
	inductive step
	\[ V_{u_j} \subseteq V(T_b), V_{u_{j + 1}} \cap V(T_b) \neq \emptyset, V_{u_{j + 1}} \cap V_{u_2} \ne \emptyset \Rightarrow V_{u_{j + 1}} \subseteq V(T_b)\]
	Therefore
	\[ V_{u_k} \cap V_{u_1} = \emptyset \]
	which contradicts $u_k u_1 \in E(g)$.

	% TODO next lecture

\end{proof}

\section{\texorpdfstring{Interval graphs}{Interval graphs}}
\vspace{5mm}
\large

If there is an interval that is \emph{open} on one side, we can replace it by smaller one but closed.
This does not change the graph.
Therefore WLOG we may assume that all intervals are closed.

Works because we have finite number of intervals.

\begin{definition}[Interval graph(INT)]
	\[ INT = IG(\{\text{interval on a line}\}) = IG(\{\text{closed intervals on a line}\}) \]
\end{definition}

\begin{theorem}[Chordal equivalent statements]
	For any graph $G$ the following are equivalent:
	\begin{enumerate}
		\item $G \in INT$
		\item $G$ allows a Clique-path decomposition
		\item $G \in IG(\{\text{subpaths of a path}\})$
	\end{enumerate}
\end{theorem}
\begin{proof}
	$1 \iff 3$
	% TODO picture

	Let $Q$ be a maximal clique of $G$.
	All intervals that are represented by $Q$ intersect.
	\[ \forall u,v \in Q: I_u \cap I_v \ne \emptyset \]
	Intervals on a line have a Helly 2-property.
	E.g. convex sets in the plain have Helly 3-property.

	Therefore
	\[ \bigcap_{u \in Q} I_n \ne \emptyset \]

	% TODO picture
	Let $x$ be the rightmost of the endpoints of $I_u, u \in Q$.
	Then $x \in \bigcap_{u \in Q} I_n$
	As otherwise
	\[ \exists v \in Q: x \notin I_v \]
	Also
	\[ \exists w \in Q: w = [x, \ldots] \Rightarrow I_v \cap I_w = \emptyset \]
	and otherwise $x$ was not the rightmost point.

	$G$ has maximal cliques: $Q = \{ Q_1, \ldots, Q_k \}$.
	Let $p_1, \ldots, p_k$ be path such that
	\[ p_i \in \bigcap_{u \in Q_i} I_u \]
	It cannot happen that $p_i = p_{i + 1}$ as it means $Q_i \cap Q_{i + 1} = p_i$.
	Combining the two will make a larger clique.
	Therefore
	\[ p_1 < \ldots < p_k \]
	Then
	\[ P = (\Q, \{ Q_i, Q_{i + 1} | i = 1, 2, \ldots, k - 1 \}) \]
	is a clique-path decomposition.

	Having
	\[ u \in Q_i, Q_h: i < h \]
	We want for $i < j < h \Rightarrow p_i, p_g \in I_u \Rightarrow p_j \in I_u \Rightarrow q \in Q_j$.

	$1 \iff 2$.

	"$\Rightarrow$"

	\[ uv \in E(G) \Rightarrow \exists i: u,v \in Q_i \Rightarrow Q_i \in V_u \cap V_v \ne \emptyset \]

	"$\Leftarrow$"
	\[ V_u \cap V_v \in Q_i \Rightarrow u,v \in Q_i \Rightarrow  uv \in E(G) \]

\end{proof}

\begin{definition}[Comparability graphs (CO)]
	POSET is $\Po = (P, \leq)$, $\leq$ is
	\begin{enumerate}
		\item antireflexive $x \not \leq x$ - no loops
		\item antisymmetric $x \leq y \Rightarrow y \not \leq x$ - no oriented 2 cycles
		\item transitive $x \leq y \land y \leq z \Rightarrow x \leq z$ - edge between 2 connected
	\end{enumerate}

	$G$ is comparability graph $\iff \exists \Po = (P, \leq)$, such that
	\[ G \simeq (P, \{ xy: x\leq y \lor y \leq x \}) := C_G(\Po) \]

	Comparing to Hasse diagram, comparability graph has more edges as in Hasse diagram we do not depict edges that come from transitivity.
\end{definition}

\begin{observation}
	How to check if graph is comparability?
	Orient every edge from the smaller to larger element.
	$G \in CO \iff$ edges of $G$ can be transitively oriented.

	Every edge gets only 1 orientation.

	Note that such orientation preserves antireflexive and antisymmetric property.
\end{observation}

\begin{notation}[Graph Complement]
	\[ -G = \left(V(G), \binom{V(G)}{2} \setminus E(G)\right) \]

	If $\A$ is a class of graphs, then $co-\A = \{ -G: G \in \A \}$.
\end{notation}

\begin{theorem}[Complements]
	.
	\begin{enumerate}
		\item $FUN = co-CO$
		\item $PER = CO \cap co-CO$
		\item $INT = CHOR \cap co-CO$
	\end{enumerate}
\end{theorem}
\begin{proof}
	$FUN \subseteq co-CO$

	Let $G \in FUN$ and $\{ c_u, u \in V(G) \}$ is a $FUN$ representation.
	Having
	\[ u \ne v: uv \ne E(G) \Rightarrow c_u \cap c_v = \emptyset \]

	From the Jordan path theorem, plane is divided into 2 parts by the line.
	$-G$ can be oriented such that $u \rightarrow v \iff c_u$ lies below $c_v$.

	Therefore either $u \rightarrow v \lor v \rightarrow u$.

	Orientation is transitive from %TODO picture.

	$co-CO \subseteq FUN^{\ast} \subseteq FUN$
	Having $G \in co-CO$ we can transitively orient non edges $\binom{V(G)}{2} \setminus E(G)$.
	Which corresponds to some partial order.

	\begin{definition}[PO dimension]
		Dimension of partial order is minimal number of linear orders $k$ such that
		\[ \exists L_1, \ldots, L_k: \Po = \bigcap L_i \]
	\end{definition}

	Take $k$ vertical lines and make $n$ points on each of the line.
	Points on the line correspond to $V(G)$.
	Put vertices in the order of linear order $L_i$ for line $i$.
	Connect the occurrences of vertex on each line.

	Take $u, v \in V(G)$ and check their lines
	\begin{gather*}
		uv \in E(G) \iff uv \notin E(-G) \Rightarrow uv \notin P \land vu \notin P \\
	\Rightarrow \exists i \ne jL uv \in L_i \land vu \in L_j \Rightarrow c_u \cap c_v \ne \emptyset
	\end{gather*}
	On the other hand
	\begin{gather*}
		wv \notin E(G) \Rightarrow uv \in E(-G) \Rightarrow uv \in \Po \lor vu \notin \Po \\
		\Rightarrow \forall i: uw \in L_i \Rightarrow c_u \cap c_w = \emptyset
	\end{gather*}
	Such piecewise continuous lines is the representation of the graph as a FUN graph.
\end{proof}
